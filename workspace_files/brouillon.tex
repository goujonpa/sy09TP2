Classification automatique

1. On remarque deux groupes en fonction de la petal length
Au final nous remarquons que deux espèces sont confondues en utilisant l'ACP.
Nous craignons maintenant qu'elles ne se fassent confondre au moment de notre
clustering

2. Sans prendre en compte les espèces, on distingue deux groupes de crabs.
Certaines parties de ces groupes sont plus concentrées, de telle sorte que l'on
peut deviner 4 éventuels groupes
En représentant par la suite l'acp en utilisant les couleurs, on distingue les
quatre groupes. Mais une fois encore, nous risquons d'avoir des difficultés à
faire des clusters

3.
On a un tableau de diss
On veut nous faire toucher du oid qu'il exite une repr qui nous permet de voir
les individus caractérisés

AFTD : décomposition du tableau de distance comme un produit de deux matrices

Méthode de sammon : même chose mais critère qe l'on optimise n'est pas le même :
points qui minimisent les distances entre les points de la repr (enfin un truc
du genre)
Méthode de cruscan : encore un autre critère : régress isotomique, on veut
respecter seulement les ordres entre distances. On veut avoir dans la
représentation finale la ossibilité de changer l'ordre


Les deux autres méthodes sont quasi les mêmes que l'AFTD donc osef un peu.
D'autant plus qu'elles n'améliorent dans notre cas que très peu la repr.

Note : si on utilise cmdscale en utilisant l'argument pour avoir les valeurs
propres-> on obtient 2à valeurs propres avc certaines négatives.

Matrice 20x20 => Espace ^20 max

> mds = cmdscale(mut, eig=T)
> str(mds)
List of 5
 $ points: num [1:20, 1:2] -6.91 -6.58 -9.85 -10.05 -9.42 ...
   ..- attr(*, "dimnames")=List of 2
     .. ..$ : chr [1:20] "Man" "Monkey" "Dog" "Horse" ...
       .. ..$ : NULL
        $ eig   : num [1:20] 7446 2262 1546 949 675 ...
         $ x     : NULL
          $ ac    : num 0
           $ GOF   : num [1:2] 0.687 0.697
           > mds$eig
            [1]  7.445705e+03  2.261504e+03  1.545843e+03  9.485851e+02
            6.750423e+02
             [6]  5.428427e+02  2.907292e+02  9.437905e+01  6.530335e+01
             3.823952e+01
             [11]  1.847890e+01  5.771593e+00  2.829321e+00  8.952838e-13
             -2.399658e+00
             [16] -1.047824e+01 -1.997258e+01 -3.400408e+01 -4.994780e+01
             -7.275051e+01
             >

Valeurs propres négatives : bizarre, difficile d'expliquer
-> limitation d'appliquer l'AFTD à un tableau qui n'est pas un tableau de
distances euclidiennes

Ici ce n'est pas un tableau de distance euclidienne donc c'est bizarre
plusieurs écoles :
- mettre à 0 négatives
- si on les mets à zéro : fausse les pourcentages d'inertie expliquée
- prendre la valeur absolue
dans le cadre de ce tp on fait ce qu'on veut
Qualité de la représentation numérique : s'effectue de la même manière que pour
l'ACP

diagramme de shepard c'est quoi ?
-> Evaluer graphiquement dij =f(delta(ij))

dist EST DIFFERENT DE as.dist
NE PAS CONFONDRE


EXE2

1. easy

2.
En gros, il n'y pas LA bonne réponse à cette question.
Il va juste falloir dire des choses intelligentes et argumentées par rapport à ce que l'on voit.
Il faut prendre en compte que l'on connait le jeu de données vite zef (comme dise les jeunes de banlieue) donc il faut aller se réferer à internet pour savoir que les iris dans le jeu de données, vienne de deux espèces normales et une troisième hybride.


3. Si on compare les résultats obtenus à la question 2 avec hclust et ceux de la question 3 avec diana, on s'aperçoit qu'il y a une légère dufférennce. En effet on distingue plus aisément, les 3 espèces avec la classification hiérarchique descendante obtenue avec diana. (hclust fait une classification ascendante) donc c'est pas pareil.


EX3

Objectif : le but de l'exercice est de tester la performance de l'algo des centres mobiles sur les trois jeux de données d'étude.

1.
utilisation de la fonction kmeans pour obtenir des partitions à la mode K appartient à {2,3,4}
> kmut = kmeans(mut, 2)
> plot(mut, col=kmut$cluster)

La fonction 







